\section{Marktanalyse}
Die meisten Apps im Play Store sollen ein Fahrtenbuch ersetzen. Diese bieten nur eingeschränkte
Funktionalitäten in der kostenlosen Version. Meist muss ein Abo abgeschlossen werden für 5-10€ pro Monat,
dafür gibt es meistens noch eine Web-Lösung und die möglichkeiten in Manipulationssichere Formate
zu expotieren. Exemplarisch werde ich ein paar Apps näher Analysieren.

\subsection{GPS Zeiterfassung + Fahrtenbuch}
Unter den Apps die ich analysiert habe wurde diese am häufigsten heruntergeladen mit über 100.000 Downloads.
Die App wurde durchschnittlich mit 4,6 Sternen bewertet bei 1.644 Rezesionen. Es gibt eine kostenlose Variante
und eine Pro Version für 9,99€.

\subsubsection{Design und Funktionalität}
Auf \ref{img:gps1} sehen Sie den Startbildschirm der App.
In der obere Hälfte des Bildschirm befinden sich Buttons \enquote{Abfahrt} und \enquote{Ankunft} um eine
Aufzeichnung zu starten und zu beende. Es gibt einen Button um den Kilometerstand des Autos zu korrigiert.
Außerdem kann man Kosten für das Auto und wie viel man getankt festhalten, was für meine App aber eher
uninteressant ist. In der unteren Hälfte des Bildschirm befindet sich eine Liste der erfassten fahrten
und rechts unten ein optionen Button (siehe \ref{img:gps3}). Wenn man auf eine erfasste fahrt klickt, siehe \ref{img:gps2},
werden neue Buttons in der oberen hälfte des Bildschirms geladen. Man hat die Möglichkeit die Fahrt als
\enquote{Dienst}, \enquote{Privat} oder \enquote{Sonst.} zu markieren, den Favoriten hinzuzufügen,
den Eintrag löschen, Fahrt auf einer Karte anzuzeigen (siehe \ref{img:gps4}) oder das hinzuzufügen von Notizen
oder eine Trennlinie in der Listen ansicht.

\begin{figure}[H]%zwei bilder nebeneinander
    \begin{minipage}[b]{.4\linewidth} % [b] => Ausrichtung an \caption
        \includegraphics[scale=0.14]{img/gps1}
        \caption{\label{img:img/gps1}GPS Z+F Startbildschirm.}
    \end{minipage}
    \hspace{0.1\linewidth}% Abstand zwischen Bilder
    \begin{minipage}[b]{.4\linewidth} % [b] => Ausrichtung an \caption
        \includegraphics[scale=0.14]{img/gps2}
        \caption{\label{img:img/gps2}GPS Z+F Fahrt ausgewählt.}
    \end{minipage}
\end{figure}

\begin{figure}[H]%zwei bilder nebeneinander
    \begin{minipage}[b]{.4\linewidth} % [b] => Ausrichtung an \caption
        \includegraphics[scale=0.14]{img/gps3}
        \caption{\label{img:img/gps3}GPS Z+F Einstellungen.}
    \end{minipage}
    \hspace{0.1\linewidth}% Abstand zwischen Bilder
    \begin{minipage}[b]{.4\linewidth} % [b] => Ausrichtung an \caption
        \includegraphics[scale=0.14]{img/gps4}
        \caption{\label{img:img/gps4}GPS Z+F aufgezeichnete Fahrt auf Karte.}
    \end{minipage}
\end{figure}

\subsubsection{Kommentare im Play Store}
\begin{enumerate}
    \item Der automatische Start und Ende der Aufzeichnungen ist sehr gut.
    \item Nicht Finanzamt konform da Änderungen nicht nachvollziehbar sind (nur in der Pro Version).
    \item Bedienung nicht induitiv.
\end{enumerate}

\subsubsection{Fazit}
Diese App ist im punkte Funktionalität woll die beste Lösung die es ohne Abo gibt. Aber designtechnisch
sehe ich noch Verbesserungspotential. Die Hälfte des Bildschirms besteht nur aus Buttons, die Liste
der erfassten Fahrten ist sehr klein und der optionen Button im rechten unteren Eck entspricht nicht
den Style-Guides. Außerdem wird der Eintrag sofort gelöscht wenn man auf löschen klickt. Gut ist die
Ansicht der Routen auf einer Karte.

\subsection{kfz fahrtenbuch}
Diese App wurde über 10.000 mal heruntergeladen und befindet sich damit im mittelfeld von den von mir
analysierten Apps. Durchschnittlich wurde die App mit 2,9 Sternen bei 52 Rezessionen bewertet.
Diese App wird in Verbindung mit einer Cloud Lösung mit einer Web Anwendung angeboten und ist mit einem
monatlichen Abo verbunden. Wenn man die kostenlose Version benutzt kann man nur Fahrten per Hand eintragen.

\subsubsection{Design und Funktionalität}
Auf \ref{img:img/kfz1} sehen Sie den Startbildschirm. Auf diesem gibt es drei Buttons, \textit{Fahrt aufzeichnen},
\textit{Fahrt erfassen} und \textit{Fahrten bearbeiten}. Die möglichkeit eine Fahr aufzuzeichnen gibt es nur
wenn man ein Abo abgeschlossen hat. \ref{img:img/kfz2} zeigt einen Scrennshot aus dem Play Store wie die Oberfläche
einer solchen Aufzeichnung aussieht. Da ich keine möglichkeite hatte die Funktion selbst zu testen werde ich hier nicht
näher drauf eingehen. Klickt man auf den Button \textit{Fahrt erfassen} sieht man die Oberfläche aus \ref{img:img/kfz3}.
Ganz Oben kann man die Art der Fahr auswählen. In den underen Zeilen kann man Start-, Zieladresse, Distanz und Kilometerstand
eingeben. Klickt man auf dem Startbildschirm auf \textit{Fahrten bearbeiten} kommt man zum Verlauf (siehe \ref{img:img/kfz4}).
Hier gibt es eine wie ein Zeitstrahl aufgebaute Ansicht aller fahrten. Bei Bedarf kann man erfasste Fahrten vervollständigen,
dazu öffnet sich wieder die selbe Oberfläche wie zum erfassen einer Fahrt.

\begin{figure}[H]%zwei bilder nebeneinander
    \begin{minipage}[b]{.4\linewidth} % [b] => Ausrichtung an \caption
        \includegraphics[scale=0.14]{img/kfz1}
        \caption{\label{img:img/kfz1}kfz Fahrtenbuch: Startbildschirm}
    \end{minipage}
    \hspace{0.1\linewidth}% Abstand zwischen Bilder
    \begin{minipage}[b]{.4\linewidth} % [b] => Ausrichtung an \caption
        \includegraphics[scale=0.17]{img/kfz2}
        \caption{\label{img:img/kfz2}kfz Fahrtenbuch: fahrt aufzeichnen}
    \end{minipage}
\end{figure}

\begin{figure}[H]%zwei bilder nebeneinander
    \begin{minipage}[b]{.4\linewidth} % [b] => Ausrichtung an \caption
        \includegraphics[scale=0.14]{img/kfz3}
        \caption{\label{img:img/kfz3}kfz Fahrtenbuch: fahrt erfassen}
    \end{minipage}
    \hspace{0.1\linewidth}% Abstand zwischen Bilder
    \begin{minipage}[b]{.4\linewidth} % [b] => Ausrichtung an \caption
        \includegraphics[scale=0.14]{img/kfz4}
        \caption{\label{img:img/kfz4}kfz Fahrtenbuch: fahrt bearbeiten}
    \end{minipage}
\end{figure}

\subsubsection{Kommentare im Play Store}
Pries Leistung Verhältnis der Cloud LÖsung wird kritisiert
\begin{enumerate}
    \item
\end{enumerate}

\subsubsection{Fazit}
Die Funktionalität in der kostenlosen Version ist sehr eingeschränkt und bietet nicht einmal die
Möglichkeit der Aufzeichnung durch GPS. Das Design der Oberfläche ist moderner und übersichtlicher
im Vergleich zu der ersten App. ABer wenn man die eingeschränkte Funktionalität in der kostenlosen
Version berücksichtigt halte ich diese App im Vergleich für die schlechteste, was auch die Bewertungen
im Play Store wiederspiegeln.

\subsection{SquareTrip}
Square Trip hat durchschnittlich 4,1 Sterne bei 118 Rezesionen bekommen und wurde über 10.000
mal heruntergeladen. Es gibt eine kostenlose und Pro Version.


\subsubsection{Design und Funktionalität}
\ref{img:img/squ1} zeigt den Startbildschirm der App.


\begin{figure}[H]%zwei bilder nebeneinander
    \begin{minipage}[b]{.4\linewidth} % [b] => Ausrichtung an \caption
        \includegraphics[scale=0.14]{img/squ1}
        \caption{\label{img:img/squ1}GPS Z+F Startbildschirm.}
    \end{minipage}
    \hspace{0.1\linewidth}% Abstand zwischen Bilder
    \begin{minipage}[b]{.4\linewidth} % [b] => Ausrichtung an \caption
        \includegraphics[scale=0.14]{img/squ2}
        \caption{\label{img:img/squ2}GPS Z+F Fahrt ausgewählt.}
    \end{minipage}
\end{figure}

\begin{figure}[H]%zwei bilder nebeneinander
    \begin{minipage}[b]{.4\linewidth} % [b] => Ausrichtung an \caption
        \includegraphics[scale=0.14]{img/squ3}
        \caption{\label{img:img/squ3}GPS Z+F Einstellungen.}
    \end{minipage}
    \hspace{0.1\linewidth}% Abstand zwischen Bilder
    \begin{minipage}[b]{.4\linewidth} % [b] => Ausrichtung an \caption
        \includegraphics[scale=0.14]{img/squ4}
        \caption{\label{img:img/squ4}GPS Z+F aufgezeichnete Fahrt auf Karte.}
    \end{minipage}
\end{figure}

\subsubsection{Kommentare im Play Store}
\begin{enumerate}
    \item Die meisten Bewertungen sind zufrieden mit der App
\end{enumerate}

\subsubsection{Fazit}
Die App läuft ununterbrochen im Hintergrund und sucht nach einem GPS SIgnal zum automatischen aufzeichnen,
was sich negativ auf den Akku auswirkt.

\subsection{Driverslog Pro 2 - Fahrtenbuch}
Nur eine Testlizenz
App ist abgestürzt und nicht mehr gestartet
durschnittliche Bewerung von 3,9 Sternen bei 60 Rezensionen und 5.000+ Downnloads.

\subsubsection{Design und Funktionalität}

\begin{figure}[H]%zwei bilder nebeneinander
    \begin{minipage}[b]{.4\linewidth} % [b] => Ausrichtung an \caption
        \includegraphics[scale=0.14]{img/pro1}
        \caption{\label{img:img/pro1}GPS Z+F Startbildschirm.}
    \end{minipage}
    \hspace{0.1\linewidth}% Abstand zwischen Bilder
    \begin{minipage}[b]{.4\linewidth} % [b] => Ausrichtung an \caption
        \includegraphics[scale=0.14]{img/pro2}
        \caption{\label{img:img/pro2}GPS Z+F Fahrt ausgewählt.}
    \end{minipage}
\end{figure}

\begin{figure}[H]%zwei bilder nebeneinander
    \begin{minipage}[b]{.4\linewidth} % [b] => Ausrichtung an \caption
        \includegraphics[scale=0.14]{img/pro3}
        \caption{\label{img:img/pro3}GPS Z+F Einstellungen.}
    \end{minipage}
    \hspace{0.1\linewidth}% Abstand zwischen Bilder
    \begin{minipage}[b]{.4\linewidth} % [b] => Ausrichtung an \caption
        \includegraphics[scale=0.14]{img/pro4}
        \caption{\label{img:img/pro4}GPS Z+F aufgezeichnete Fahrt auf Karte.}
    \end{minipage}
\end{figure}


\begin{figure}[H]
        \includegraphics[scale=0.14]{img/pro5}
        \caption{\label{img:img/pro5}GPS Z+F Einstellungen.}
\end{figure}


\subsubsection{Kommentare im Play Store}
\begin{enumerate}
    \item
\end{enumerate}

\subsubsection{Fazit}

\subsection{Fazit}
Generell lässt sich sagen das es im Play Store eher wenige Apps gibt.
