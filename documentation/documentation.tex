\documentclass{article}

\usepackage[T1]{fontenc}
\usepackage{newtxtext}     %TimesNewRoman
\usepackage[utf8]{inputenc} %UTF-8 Zeichensatz
\usepackage[ngerman]{babel} %Deutsch
\usepackage{lmodern}
\usepackage{pdfpages}
\usepackage{graphicx}

%Allgemeine Angaben
\title{Entwicklung Mobiler Anwendungen \\Projekt logbook \\ Dokumentation}
\author{Jannik Ostermayer}
%\date{}

\begin{document}

\begin{titlepage}
    \clearpage
    \maketitle          %allgemeine Angaben
    \thispagestyle{empty}
\end{titlepage}

\tableofcontents    %Inhaltsverzeichnis
\pagebreak          %Seitenumbruch

\section{Idee}
Meine Idee ist eine Fahrtenbuchapp zum leichten Übertragen 
der Fahrten in ein physisches Fahrtenbuch.
Da es noch keine verifizierte Fahrtenbuchapp 
des Finanzamtes gibt und so auch keine Rechtssicherheit herrscht.
Zum leichteren übertragen sollte es auch eine Exportfunktion geben.
Da das übertragen in ein Fahrtenbuch zeitnahgeschehen (innerhalb der 
nächsten drei Tage) erfolgen muss sollte es auch Push Benachrichtungen
zur Erinnerung geben. 

\section{Beschreibung}
Diese App vereinfacht das führen eines Fahrtenbuches enorm. 
Alle Fahrten werden erfasst und du wirst Benachrichtigt, dass du dein 
Fahrtenahrtenbuch zeitnah führen kannst. 
Außerdem gibt es Export-Funktionen das du dir deine erfassetn Fahrten 
auch bequem auf deinem PC anschauen kannst.

\section{Zielgruppe}
Der potentielle Markt für eine Fahrtenbuch App erstreckt sich potentiell über alle
gewerblich genutzt Fahrzeuge in Deutschland. Das sind laut einer Statistik des KBA (Kraft Fahrt Bundesamtes)
aus dem Jahr 2017 64,4% der zugelassenen Fahrzeuge in Deutschland. In absoluten Zahlen sind das 2.215.208
Gewerblich neu zugelassen Fahrzeuge. Meine App richtet sich aber spezial an Personen die einen
Firmenwagen fahren, wie Vertreter. Da es bei den Statistiken nur eine unterscheidung zwischen
geschäftlicher und privater Nutzung gibt, exestieren keine genauen Zahlen über Anzahl der Firmenwagen
unter den gewerblich genutzten Fahrzeugen. Man geht von ca 10% der gesamten Fahrzeuge aus.
Laut KBA vom Stand 1. Januar 2018 gibt es in Deutschland 46,5 Millionen zugelessene PKWs.
Das bedeutet wenn man von 10% ausgeht, gibt es in Deutschland 4,65 Millionen Firmenfahrzeuge.
Das sich meine App an Fahrer eines Firmenwagens richtet die Steuern sparen wollen und eine
erleichterung beim führen eines Fahrtenbuches suchen, trifft meine App auf einen potentiellen
Markt von 4,65 Millionen nutzern.

\begin{figure}[h]
  \centering
     \includegraphics[width=0.7\textwidth]{Firmenwagen_Branchen.PNG}
  \caption{Firmenwagen nach Branchen}
  \label{fig:Bild1}
\end{figure}


\section{Marktanalyse}
So gut wie alles Kostenpflichtig.
1. kfz fahrtenbuch
2. gps zeiterfassung

\section{Bedarfsanalyse}

\section{Features}
\subsection{Anforderungen}
	\begin{enumerate}
	\item Die Fahrtenbuch Daten sollen in einer Datenbank auf dem Handy gespeichert werden.
	\item Alle Einträge der Datenbank sollen in einer Liste dargestellt werden.
	\item Die Liste enthält die Information Zeit, Zweck, Start, Ziel
	\item Die Liste kann man nach Datum sotieren.
	\item Der Benutzer kann in der Liste suchen.
	\item Der Benutzer kann auf ein Element der Liste klicken und bekommt eine Detailansicht.
	\item Die App speichert die Daten in der Form wir die vom Finanzamt anerkannten Fahrtenbücher.
	\item Die App muss per GPS erfassen wenn der Benutzer losfährt und automatisch Start-, Endpunkt und Entfernung bestimmen und diese eintragen.
	\item Die App muss den Benutzer Auffordern fehlende Information, wie "Zweck der Fahrt nachzutragen".
	\item Der Benutzer muss Fahrten in der App händisch eintragen können.
	\item Der Benutzer muss alle Fahrtenbuch Daten expotieren können als PDF oder HTML(bzw. Jason).
	\item Die App soll automatisch starten wenn das Handy per Blutooth mit dem Auto gekoppelt wird.
	\item Der Benutzer soll Einträge bearbeiten können.
	\item Der Benutzer soll Einträge löschen können.
	\end{enumerate}

\section{User Stories, Personas}

\section{Mockuop, Wireframe}

\section{UML-Diagramme}

\section{Code Dokumentation}

\section{Fazit}
%Input(einleitung.tex) ->einleitung.tex einbinden

\end{document}