\documentclass{article}

\usepackage[T1]{fontenc}
\usepackage{newtxtext}     %TimesNewRoman
\usepackage[utf8]{inputenc} %UTF-8 Zeichensatz
\usepackage[ngerman]{babel} %Deutsch
\usepackage{lmodern}

%Allgemeine Angaben
\title{Entwicklung Mobiler Anwendungen}
\author{Jannik Ostermayer}
%\date{}

\begin{document}

\begin{titlepage}
    \clearpage
    \maketitle          %allgemeine Angaben
    \thispagestyle{empty}
\end{titlepage}

\tableofcontents    %Inhaltsverzeichnis
\pagebreak          %Seitenumbruch

\section{Idee}
Meine Idee ist eine Fahrtenbuchapp die einfaches Übertragen der erfassten Daten in ein Fahrtenbuch ermöglicht.
Da es noch keine verifizierte Fahrtenbuchapp des Finanzamtes gibt und so auch keine Rechtsicherheit.
Außerdem sollte es eine einfache Möglichkeit geben die Daten zu expotieren, um sie zum Beispiel auf dem
PC anzuzeigen, für ein einfaches übertragen in das Fahrtenbuch.

\section{Beschreibung}

\section{Zielgruppe}

\section{Marktanalyse}
So gut wie alles Kostenpflichtig.
1. kfz fahrtenbuch
2. gps zeiterfassung

\section{Bedarfsanalyse}

\section{Features}

\section{User Stories, Personas}

\section{Mockuop, Wireframe}

\section{UML-Diagramme}

\section{Code Dokumentation}

\section{Fazit}
%Input(einleitung.tex) ->einleitung.tex einbinden

\end{document}