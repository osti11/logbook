\documentclass{article}

\usepackage[T1]{fontenc}
\usepackage{newtxtext}     %TimesNewRoman
\usepackage[utf8]{inputenc} %UTF-8 Zeichensatz
\usepackage[ngerman]{babel} %Deutsch
\usepackage{lmodern}
\usepackage{pdfpages}

%Allgemeine Angaben
\title{Entwicklung Mobiler Anwendungen \\Projekt logbook \\ Dokumentation}
\author{Jannik Ostermayer}
%\date{}

\begin{document}

\begin{titlepage}
    \clearpage
    \maketitle          %allgemeine Angaben
    \thispagestyle{empty}
\end{titlepage}

\tableofcontents    %Inhaltsverzeichnis
\pagebreak          %Seitenumbruch

\section{Idee}
Meine Idee ist eine Fahrtenbuchapp die einfaches Übertragen der erfassten Daten in ein Fahrtenbuch ermöglicht.
Da es noch keine verifizierte Fahrtenbuchapp des Finanzamtes gibt und so auch keine Rechtsicherheit.
Außerdem sollte es eine einfache Möglichkeit geben die Daten zu expotieren, um sie zum Beispiel auf dem
PC anzuzeigen, für ein einfaches übertragen in das Fahrtenbuch.

\section{Beschreibung}

\section{Zielgruppe}

\section{Marktanalyse}
So gut wie alles Kostenpflichtig.
1. kfz fahrtenbuch
2. gps zeiterfassung

\section{Bedarfsanalyse}

\section{Features}
\subsection{Anforderungen}
	\begin{enumerate}
	\item Die Fahrtenbuch Daten sollen in einer Datenbank auf dem Handy gespeichert werden.
	\item Alle Einträge der Datenbank sollen in einer Liste dargestellt werden.
	\item Die Liste enthält die Information Zeit, Zweck, Start, Ziel
	\item Die Liste kann man nach Datum sotieren.
	\item Der Benutzer kann in der Liste suchen.
	\item Der Benutzer kann auf ein Element der Liste klicken und bekommt eine Detailansicht.
	\item Die App speichert die Daten in der Form wir die vom Finanzamt anerkannten Fahrtenbücher.
	\item Die App muss per GPS erfassen wenn der Benutzer losfährt und automatisch Start-, Endpunkt und Entfernung bestimmen und diese eintragen.
	\item Die App muss den Benutzer Auffordern fehlende Information, wie "Zweck der Fahrt nachzutragen".
	\item Der Benutzer muss Fahrten in der App händisch eintragen können.
	\item Der Benutzer muss alle Fahrtenbuch Daten expotieren können als PDF oder HTML(bzw. Jason).
	\item Die App soll automatisch starten wenn das Handy per Blutooth mit dem Auto gekoppelt wird.
	\item Der Benutzer soll Einträge bearbeiten können.
	\item Der Benutzer soll Einträge löschen können.
	\end{enumerate}

\section{User Stories, Personas}

\section{Mockuop, Wireframe}

\section{UML-Diagramme}

\section{Code Dokumentation}

\section{Fazit}
%Input(einleitung.tex) ->einleitung.tex einbinden

\end{document}