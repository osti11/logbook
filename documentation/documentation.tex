\documentclass[a4paper]{article}

\usepackage[T1]{fontenc}
\usepackage{newtxtext}     %TimesNewRoman
\usepackage[utf8]{inputenc} %UTF-8 Zeichensatz
\usepackage[ngerman]{babel} %Deutsch
\usepackage{lmodern}
\usepackage{pdfpages}
\usepackage{graphicx}   %Grafik einfuegen
\usepackage{float}      %Grafik fest positionieren
\usepackage{scrlayer-scrpage}   %Paket Kopf-/Fußzeile
\usepackage{csquotes}
%\setcounter{tocdepth}{2}

\author{Jannik Ostermayer}

\begin{document}

\pagestyle{scrheadings}         %Kopf und Fußzeile mit Paket setzen
\clearpairofpagestyles          %Kopf & Fuß leeren
%footer
\ifoot{Jannik Ostermayer}
\cfoot{Matr.Nr. 672330}
\ofoot{\pagemark}

\begin{titlepage}

	\centering
	\includegraphics[width=0.4\textwidth]{img/Logo_HS_Worms.png}\par\vspace{1cm}
	{\LARGE Entwicklung mobiler Anwendungen \par}
	\vspace{1.5cm}
	{\huge Projekt Fahrtenbuch\par}
	\vspace{0.3cm}
	{\huge Dokumentation\par}
	\vspace{2cm}
	{\Large Jannik Ostermayer\par}
	\vspace{1cm}
	{ Matrikelnummer 672330 \par}
	\vfill

% Bottom of the page
	{\large \today\par}

\end{titlepage}

\tableofcontents    %Inhaltsverzeichnis
\pagebreak          %Seitenumbruch

\section{Idee}
Meine Idee ist eine Fahrtenbuchapp zum leichten Übertragen 
der Fahrten in ein physisches Fahrtenbuch.
Da es noch keine verifizierte Fahrtenbuchapp 
des Finanzamtes gibt und so auch keine Rechtssicherheit herrscht.
Zum leichteren übertragen sollte es auch eine Exportfunktion geben oder
die Möglichleit sich eine Email zu schicken.
Da das übertragen in ein Fahrtenbuch zeitnahgeschehen (innerhalb der 
nächsten drei Tage) erfolgen muss sollte es auch Push Benachrichtungen
zur Erinnerung geben. Die App kann durch einen Button gestartet werden,
aber auch alternativ durch eine Bluetooth Verbindung oder einen NFC Chip.
Die erfassten Fahrten sollten auch auf einer Karte angezeigt werden,
da man bei größeren Umwegen, diese genau beschreiben muss.

\section{Beschreibung}
Diese App vereinfacht das führen eines Fahrtenbuches enorm. 
Alle Fahrten werden erfasst und du wirst Benachrichtigt, dass du dein 
Fahrtenahrtenbuch zeitnah führen kannst. 
Außerdem gibt es Export-Funktionen das du dir deine erfassetn Fahrten 
auch bequem auf deinem PC anschauen kannst.

\section{Zielgruppe}
Der potentielle Markt für eine Fahrtenbuch App erstreckt sich über alle
gewerblich genutzt Fahrzeuge in Deutschland. Das sind laut einer Statistik des KBA (Kraft Fahrt Bundesamtes)
aus dem Jahr 2017 64,4\% der zugelassenen Fahrzeuge in Deutschland. In absoluten Zahlen sind das 2.215.208
Gewerblich neu zugelassen Fahrzeuge. Meine App richtet sich aber spezial an Personen die einen
Firmenwagen fahren, wie Vertreter. Da es bei den Statistiken nur eine unterscheidung zwischen
geschäftlicher und privater Nutzung gibt, exestieren keine genauen Zahlen über Anzahl der Firmenwagen
unter den gewerblich genutzten Fahrzeugen. Man geht von ca 10\% der gesamten Fahrzeuge aus.
Laut KBA vom Stand 1. Januar 2018 gibt es in Deutschland 46,5 Millionen zugelessene PKWs.
Das bedeutet wenn man von 10\% ausgeht, gibt es in Deutschland 4,65 Millionen Firmenfahrzeuge.
Da sich meine App an Fahrer eines Firmenwagens richtet die Steuern sparen wollen und eine
erleichterung beim führen eines Fahrtenbuches suchen, trifft meine App auf einen potentiellen
Markt von 4,65 Millionen nutzern. Bei dem Führen von Fahrtenbüchern gibt es starke unterschiede
zwischen den Berufsgruppen, so muss zum Beispiel ein Handwerker für jede Fahrt den Grund der Fahrt angeben.
Dahingegen müssen Handelsvertreter und Kundendienstmonteure keinen Grund der Fahrt angeben, wenn der Grund
plausibel ist, hier reicht wenn der Kunde angegeben wird. Bei diesen Berufsgruppen die nicht alles angeben
müssen spricht man von einem \textit{erleichterten} Fahrtenbuch. In erste Linie soll sich meine App an Personen
die ein solches \textit{erleichtertes} Fahrtenbuch führen richten. Im nächsten Schritt, wenn meine App ausgebaut wird
würde ich auch gerne die anderen Berufsgruppen, die ein \textit{normales} Fahrtenbuch führen in
meine Zielgruppe mitaufnehmen.

%Grafik
%\begin{figure}[H]
%    \begin{center}
%        \includegraphics[width=12cm]{neuzulassung.jpg}
%        \caption{Verteilung der neuzugelassenen Fahrzeuge}
%        \label{Verteilung der neuzugelassenen Fahrzeuge}
%    \end{center}
%\end{figure}

\section{Marktanalyse}
Die meisten Apps im Play Store sollen ein Fahrtenbuch ersetzen. Diese bieten nur eingeschränkte
Funktionalitäten in der kostenlosen Version. Meist muss ein Abo abgeschlossen werden für 5-10€ pro Monat,
dafür gibt es meistens noch eine Web-Lösung und die möglichkeiten in Manipulationssichere Formate
zu expotieren. Exemplarisch werde ich ein paar Apps näher Analysieren.

\subsection{GPS Zeiterfassung + Fahrtenbuch}
Unter den Apps die ich analysiert habe wurde diese am häufigsten heruntergeladen mit über 100.000 Downloads.
Die App wurde durchschnittlich mit 4,6 Sternen bewertet bei 1.644 Rezesionen. Es gibt eine kostenlose Variante
und eine Pro Version für 9,99€.

\subsubsection{Design und Funktionalität}
Auf \ref{img:gps1} sehen Sie den Startbildschirm der App.
In der obere Hälfte des Bildschirm befinden sich Buttons \enquote{Abfahrt} und \enquote{Ankunft} um eine
Aufzeichnung zu starten und zu beende. Es gibt einen Button um den Kilometerstand des Autos zu korrigiert.
Außerdem kann man Kosten für das Auto und wie viel man getankt festhalten, was für meine App aber eher
uninteressant ist. In der unteren Hälfte des Bildschirm befindet sich eine Liste der erfassten fahrten
und rechts unten ein optionen Button (siehe \ref{img:gps3}). Wenn man auf eine erfasste fahrt klickt, siehe \ref{img:gps2},
werden neue Buttons in der oberen hälfte des Bildschirms geladen. Man hat die Möglichkeit die Fahrt als
\enquote{Dienst}, \enquote{Privat} oder \enquote{Sonst.} zu markieren, den Favoriten hinzuzufügen,
den Eintrag löschen, Fahrt auf einer Karte anzuzeigen (siehe \ref{img:gps4}) oder das hinzuzufügen von Notizen
oder eine Trennlinie in der Listen ansicht.

\begin{figure}[H]%zwei bilder nebeneinander
    \begin{minipage}[b]{.4\linewidth} % [b] => Ausrichtung an \caption
        \includegraphics[scale=0.14]{img/gps1}
        \caption{\label{img:img/gps1}GPS Z+F Startbildschirm.}
    \end{minipage}
    \hspace{0.1\linewidth}% Abstand zwischen Bilder
    \begin{minipage}[b]{.4\linewidth} % [b] => Ausrichtung an \caption
        \includegraphics[scale=0.14]{img/gps2}
        \caption{\label{img:img/gps2}GPS Z+F Fahrt ausgewählt.}
    \end{minipage}
\end{figure}

\begin{figure}[H]%zwei bilder nebeneinander
    \begin{minipage}[b]{.4\linewidth} % [b] => Ausrichtung an \caption
        \includegraphics[scale=0.14]{img/gps3}
        \caption{\label{img:img/gps3}GPS Z+F Einstellungen.}
    \end{minipage}
    \hspace{0.1\linewidth}% Abstand zwischen Bilder
    \begin{minipage}[b]{.4\linewidth} % [b] => Ausrichtung an \caption
        \includegraphics[scale=0.14]{img/gps4}
        \caption{\label{img:img/gps4}GPS Z+F aufgezeichnete Fahrt auf Karte.}
    \end{minipage}
\end{figure}

\subsubsection{Kommentare im Play Store}
\begin{enumerate}
    \item Der automatische Start und Ende der Aufzeichnungen ist sehr gut.
    \item Nicht Finanzamt konform da Änderungen nicht nachvollziehbar sind (nur in der Pro Version).
    \item Bedienung nicht induitiv.
\end{enumerate}

\subsubsection{Fazit}
Diese App ist im punkte Funktionalität woll die beste Lösung die es ohne Abo gibt. Aber designtechnisch
sehe ich noch Verbesserungspotential. Die Hälfte des Bildschirms besteht nur aus Buttons, die Liste
der erfassten Fahrten ist sehr klein und der optionen Button im rechten unteren Eck entspricht nicht
den Style-Guides. Außerdem wird der Eintrag sofort gelöscht wenn man auf löschen klickt. Gut ist die
Ansicht der Routen auf einer Karte.

\subsection{kfz fahrtenbuch}
Diese App wurde über 10.000 mal heruntergeladen und befindet sich damit im mittelfeld von den von mir
analysierten Apps. Durchschnittlich wurde die App mit 2,9 Sternen bei 52 Rezessionen bewertet.
Diese App wird in Verbindung mit einer Cloud Lösung mit einer Web Anwendung angeboten und ist mit einem
monatlichen Abo verbunden. Wenn man die kostenlose Version benutzt kann man nur Fahrten per Hand eintragen.

\subsubsection{Design und Funktionalität}
Auf \ref{img:img/kfz1} sehen Sie den Startbildschirm. Auf diesem gibt es drei Buttons, \textit{Fahrt aufzeichnen},
\textit{Fahrt erfassen} und \textit{Fahrten bearbeiten}. Die möglichkeit eine Fahr aufzuzeichnen gibt es nur
wenn man ein Abo abgeschlossen hat. \ref{img:img/kfz2} zeigt einen Scrennshot aus dem Play Store wie die Oberfläche
einer solchen Aufzeichnung aussieht. Da ich keine möglichkeite hatte die Funktion selbst zu testen werde ich hier nicht
näher drauf eingehen. Klickt man auf den Button \textit{Fahrt erfassen} sieht man die Oberfläche aus \ref{img:img/kfz3}.
Ganz Oben kann man die Art der Fahr auswählen. In den underen Zeilen kann man Start-, Zieladresse, Distanz und Kilometerstand
eingeben. Klickt man auf dem Startbildschirm auf \textit{Fahrten bearbeiten} kommt man zum Verlauf (siehe \ref{img:img/kfz4}).
Hier gibt es eine wie ein Zeitstrahl aufgebaute Ansicht aller fahrten. Bei Bedarf kann man erfasste Fahrten vervollständigen,
dazu öffnet sich wieder die selbe Oberfläche wie zum erfassen einer Fahrt.

\begin{figure}[H]%zwei bilder nebeneinander
    \begin{minipage}[b]{.4\linewidth} % [b] => Ausrichtung an \caption
        \includegraphics[scale=0.14]{img/kfz1}
        \caption{\label{img:img/kfz1}kfz Fahrtenbuch: Startbildschirm}
    \end{minipage}
    \hspace{0.1\linewidth}% Abstand zwischen Bilder
    \begin{minipage}[b]{.4\linewidth} % [b] => Ausrichtung an \caption
        \includegraphics[scale=0.17]{img/kfz2}
        \caption{\label{img:img/kfz2}kfz Fahrtenbuch: fahrt aufzeichnen}
    \end{minipage}
\end{figure}

\begin{figure}[H]%zwei bilder nebeneinander
    \begin{minipage}[b]{.4\linewidth} % [b] => Ausrichtung an \caption
        \includegraphics[scale=0.14]{img/kfz3}
        \caption{\label{img:img/kfz3}kfz Fahrtenbuch: fahrt erfassen}
    \end{minipage}
    \hspace{0.1\linewidth}% Abstand zwischen Bilder
    \begin{minipage}[b]{.4\linewidth} % [b] => Ausrichtung an \caption
        \includegraphics[scale=0.14]{img/kfz4}
        \caption{\label{img:img/kfz4}kfz Fahrtenbuch: fahrt bearbeiten}
    \end{minipage}
\end{figure}

\subsubsection{Kommentare im Play Store}
Pries Leistung Verhältnis der Cloud LÖsung wird kritisiert
\begin{enumerate}
    \item
\end{enumerate}

\subsubsection{Fazit}
Die Funktionalität in der kostenlosen Version ist sehr eingeschränkt und bietet nicht einmal die
Möglichkeit der Aufzeichnung durch GPS. Das Design der Oberfläche ist moderner und übersichtlicher
im Vergleich zu der ersten App. ABer wenn man die eingeschränkte Funktionalität in der kostenlosen
Version berücksichtigt halte ich diese App im Vergleich für die schlechteste, was auch die Bewertungen
im Play Store wiederspiegeln.

\subsection{SquareTrip}
Square Trip hat durchschnittlich 4,1 Sterne bei 118 Rezesionen bekommen und wurde über 10.000
mal heruntergeladen. Es gibt eine kostenlose und Pro Version.


\subsubsection{Design und Funktionalität}
\ref{img:img/squ1} zeigt den Startbildschirm der App.


\begin{figure}[H]%zwei bilder nebeneinander
    \begin{minipage}[b]{.4\linewidth} % [b] => Ausrichtung an \caption
        \includegraphics[scale=0.14]{img/squ1}
        \caption{\label{img:img/squ1}GPS Z+F Startbildschirm.}
    \end{minipage}
    \hspace{0.1\linewidth}% Abstand zwischen Bilder
    \begin{minipage}[b]{.4\linewidth} % [b] => Ausrichtung an \caption
        \includegraphics[scale=0.14]{img/squ2}
        \caption{\label{img:img/squ2}GPS Z+F Fahrt ausgewählt.}
    \end{minipage}
\end{figure}

\begin{figure}[H]%zwei bilder nebeneinander
    \begin{minipage}[b]{.4\linewidth} % [b] => Ausrichtung an \caption
        \includegraphics[scale=0.14]{img/squ3}
        \caption{\label{img:img/squ3}GPS Z+F Einstellungen.}
    \end{minipage}
    \hspace{0.1\linewidth}% Abstand zwischen Bilder
    \begin{minipage}[b]{.4\linewidth} % [b] => Ausrichtung an \caption
        \includegraphics[scale=0.14]{img/squ4}
        \caption{\label{img:img/squ4}GPS Z+F aufgezeichnete Fahrt auf Karte.}
    \end{minipage}
\end{figure}

\subsubsection{Kommentare im Play Store}
\begin{enumerate}
    \item Die meisten Bewertungen sind zufrieden mit der App
\end{enumerate}

\subsubsection{Fazit}
Die App läuft ununterbrochen im Hintergrund und sucht nach einem GPS SIgnal zum automatischen aufzeichnen,
was sich negativ auf den Akku auswirkt.

\subsection{Driverslog Pro 2 - Fahrtenbuch}
Nur eine Testlizenz
App ist abgestürzt und nicht mehr gestartet
durschnittliche Bewerung von 3,9 Sternen bei 60 Rezensionen und 5.000+ Downnloads.

\subsubsection{Design und Funktionalität}

\begin{figure}[H]%zwei bilder nebeneinander
    \begin{minipage}[b]{.4\linewidth} % [b] => Ausrichtung an \caption
        \includegraphics[scale=0.14]{img/pro1}
        \caption{\label{img:img/pro1}GPS Z+F Startbildschirm.}
    \end{minipage}
    \hspace{0.1\linewidth}% Abstand zwischen Bilder
    \begin{minipage}[b]{.4\linewidth} % [b] => Ausrichtung an \caption
        \includegraphics[scale=0.14]{img/pro2}
        \caption{\label{img:img/pro2}GPS Z+F Fahrt ausgewählt.}
    \end{minipage}
\end{figure}

\begin{figure}[H]%zwei bilder nebeneinander
    \begin{minipage}[b]{.4\linewidth} % [b] => Ausrichtung an \caption
        \includegraphics[scale=0.14]{img/pro3}
        \caption{\label{img:img/pro3}GPS Z+F Einstellungen.}
    \end{minipage}
    \hspace{0.1\linewidth}% Abstand zwischen Bilder
    \begin{minipage}[b]{.4\linewidth} % [b] => Ausrichtung an \caption
        \includegraphics[scale=0.14]{img/pro4}
        \caption{\label{img:img/pro4}GPS Z+F aufgezeichnete Fahrt auf Karte.}
    \end{minipage}
\end{figure}


\begin{figure}[H]
        \includegraphics[scale=0.14]{img/pro5}
        \caption{\label{img:img/pro5}GPS Z+F Einstellungen.}
\end{figure}


\subsubsection{Kommentare im Play Store}
\begin{enumerate}
    \item
\end{enumerate}

\subsubsection{Fazit}

\subsection{Fazit}
Generell lässt sich sagen das es im Play Store eher wenige Apps gibt.


\section{Bedarfsanalyse}

\subsection{Datenbank}

 \begin{table}
	\caption{Spalten des Fahrtenbuches}
    \centering
	\begin{tabular}{c c c c c c c c c c c c c}
		\hline
		Datum & Fahrzeit von - bis & Zweck der Fahrt  & Besuchte Person, Firma, Behörde &
		km-Stand Fahrtbeginn & Gefahrene km & Start & Ziel & km-Satnd Fahrtende & Ktaftstoff &
		Ltr. je 100km & Sonstiger Betrag & Name des Fahrers\\ [0.5ex]
		1 & Home & HS & lernen \\ [1ex]
		\hline
	\end{tabular}
	\label{table:nonlin}
 \end{table}

 \begin{table}
	\caption{Aufbau Datenbank}
    \centering
	\begin{tabular}{c c c c c c c c c c c c c}
		\hline
        id & start & destination & purpose & mileage_start & mileage_destination & odometer \\ [1ex]
		\hline
	\end{tabular}
	\label{table:nonlin}
 \end{table}

\subsection{GPS}

\subsubsection{Darstellung der Fahrt auf Karte}
Alle erfassten Daten sollten im GPSX Format erfasst werden. 
Diese können benutzt werden um die Route in Google Maps darzustellen.

\subsection{Bluetooth}

\subsection{View}

\subsection{Architektur}

\section{Features}

\subsection{Anforderungen}

\begin{enumerate}
	\item Die Fahrtenbuch Daten sollen in einer Datenbank auf dem Handy gespeichert werden.
	\item Alle Einträge der Datenbank sollen in einer Liste dargestellt werden.
	\item Die Liste enthält die Information Zeit, Zweck, Start, Ziel
	\item Die Liste kann man nach Datum sotieren.
	\item Der Benutzer kann in der Liste suchen.
	\item Der Benutzer kann auf ein Element der Liste klicken und bekommt eine Detailansicht.
	\item Die App speichert die Daten in der Form wir die vom Finanzamt anerkannten Fahrtenbücher.
	\item Die App muss per GPS erfassen wenn der Benutzer losfährt und automatisch Start-, Endpunkt und Entfernung bestimmen und diese eintragen.
	\item Die App muss den Benutzer Auffordern fehlende Information, wie \enquote{Zweck der Fahrt nachzutragen}.
	\item Der Benutzer muss Fahrten in der App händisch eintragen können.
	\item Der Benutzer muss alle Fahrtenbuch Daten expotieren können als PDF oder HTML(bzw. Jason).
	\item Die App soll automatisch starten wenn das Handy per Blutooth mit dem Auto gekoppelt wird.
	\item Der Benutzer soll Einträge bearbeiten können.
	\item Der Benutzer soll Einträge löschen können.
\end{enumerate}

\section{User Stories, Personas}

\section{Mockuop, Wireframe}
--erste Version--

\begin{figure}
	\begin{minipage}[b]{0.3\textwidth}
        \includegraphics[width=\textwidth]{img/mock1}
        \caption{\label{img:img/mock1}, Startbildschirm}
    \end{minipage}
    \hspace{0.025\textwidth}% Abstand zwischen Bilder
	\begin{minipage}[b]{0.3\textwidth}
        \includegraphics[width=\textwidth]{img/mock2}
        \caption{\label{img:img/mock2}Driverslog Pro 2 - Fahrtenbuch, Detailansicht}
    \end{minipage}
    \hspace{0.025\textwidth}% Abstand zwischen Bilder
	\begin{minipage}[b]{0.3\textwidth}
		\includegraphics[width=\textwidth]{img/mock3}
        \caption{\label{img:img/mock3}Driverslog Pro 2 - Fahrtenbuch, Detailansicht}
	\end{minipage}
\end{figure}

\begin{figure}
	\begin{minipage}[b]{0.3\textwidth}
        \includegraphics[width=\textwidth]{img/mock4}
        \caption{\label{img:img/mock4}, Startbildschirm}
    \end{minipage}
    \hspace{0.025\textwidth}% Abstand zwischen Bilder
	\begin{minipage}[b]{0.3\textwidth}
        \includegraphics[width=\textwidth]{img/mock5}
        \caption{\label{img:img/mock5}Driverslog Pro 2 - Fahrtenbuch, Detailansicht}
    \end{minipage}
    \hspace{0.025\textwidth}% Abstand zwischen Bilder
	\begin{minipage}[b]{0.3\textwidth}
		\includegraphics[width=\textwidth]{img/mock6}
        \caption{\label{img:img/mock6}Driverslog Pro 2 - Fahrtenbuch, Detailansicht}
	\end{minipage}
\end{figure}

\section{UML-Diagramme}

\section{Code Dokumentation}

\section{Fazit}
%Input(einleitung.tex) ->einleitung.tex einbinden

\end{document}