\documentclass{article}

\usepackage[T1]{fontenc}
\usepackage{newtxtext}     %TimesNewRoman
\usepackage[utf8]{inputenc} %UTF-8 Zeichensatz
\usepackage[ngerman]{babel} %Deutsch
\usepackage{lmodern}

%Allgemeine Angaben
\title{Entwicklung Mobiler Anwendungen}
\author{Jannik Ostermayer}
%\date{}

\begin{document}

\begin{titlepage}
    \clearpage
    \maketitle          %allgemeine Angaben
    \thispagestyle{empty}
\end{titlepage}

\tableofcontents    %Inhaltsverzeichnis
\pagebreak          %Seitenumbruch

\section{Idee}
Meine Idee ist eine Fahrtenbuchapp zum leichten Übertragen 
der Fahrten in ein physisches Fahrtenbuch.
Da es noch keine verifizierte Fahrtenbuchapp 
des Finanzamtes gibt und so auch keine Rechtssicherheit herrscht.
Zum leichteren übertragen sollte es auch eine Exportfunktion geben.
Da das übertragen in ein Fahrtenbuch zeitnahgeschehen (innerhalb der 
nächsten drei Tage) erfolgen muss sollte es auch Push Benachrichtungen
zur Erinnerung geben. 

\section{Beschreibung}
Diese App vereinfacht das führen eines Fahrtenbuches enorm. 
Alle Fahrten werden erfasst und du wirst Benachrichtigt, dass du dein 
Fahrtenahrtenbuch zeitnah führen kannst. 
Außerdem gibt es Export-Funktionen das du dir deine erfassetn Fahrten 
auch bequem auf deinem PC anschauen kannst.

\section{Zielgruppe}
Meine App richtet sich spezial an Privatpersonen die ein Firmenwagen haben,
Steuern sparen wollen und von der fehlenden Rechtssicherheit einer Fahrtenbuchapp 
abgeschreckt werden. Deshalb eine Unterstützung suchen beim führen eines Fahrtenbuchs.


\section{Marktanalyse}
So gut wie alles Kostenpflichtig.
1. kfz fahrtenbuch
2. gps zeiterfassung

\section{Bedarfsanalyse}

\section{Features}

\section{User Stories, Personas}

\section{Mockuop, Wireframe}

\section{UML-Diagramme}

\section{Code Dokumentation}

\section{Fazit}
%Input(einleitung.tex) ->einleitung.tex einbinden

\end{document}